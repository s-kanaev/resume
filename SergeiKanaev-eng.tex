% !TeX encoding = UTF-8
% !TeX root = SergeiKanaev-eng.tex
% !TeX spellcheck = en_US
\documentclass[11pt,a4paper,russian]{moderncv}

\moderncvtheme[grey]{classic}

\usepackage[T2A]{fontenc}
\usepackage[utf8]{inputenc}

\usepackage[scale=0.75]{geometry}

\name{Sergei}{Kanaev}
\email{ksergy.91@gmail.com}                               % optional, remove / comment the line if not wanted
\social[github]{ksergy}                              % optional, remove / comment the line if not wanted
\extrainfo{skype: \texttt{fazedies}}                 % optional, remove / comment the line if not wanted

\makeatletter\renewcommand*{\bibliographyitemlabel}{\@biblabel{\arabic{enumiv}}}\makeatother

\begin{document}
\makecvtitle

\section{Education}
\cventry{2008--2013}{Master's degree}{Sevastopol National University of Nuclear Energy and Industry}{Institue of Atomic Energy}{Automation of technological processes and manufactures}{}
\section{Experience}
\cventry{10.2016 --- 05.2018}{Engineer}{Brogaming Studio / Starlab Studio}{Samara}{\newline{}Mobile game server development}{During this work I was involved in: %
\begin{itemize}
\item development and implementation of test client to test some server features;
\item development and implementation of players mathchmaking;
\item development of database schema, inter-server and client-server protocol for some of server features;
\item implementation of some server features;
\item improving server robustness;
\item optimizing server;
\item \texttt{bash}/\texttt{perl} scripting for load testing;
\item some \texttt{python} testing for simplified game balance calculus;
\end{itemize}%
During this work I've had experience with \texttt{libpq}, \texttt{libsqlite3}, \texttt{libev}, \texttt{libconfig} and \texttt{jsoncpp} libraries. Also, I've provided formalized description of problem for new project in order to implement prototype application.}
\cventry{Jan 2015 --- Sep 2016}{Engineer}{Satellite Soft Labs}{Saratov}{\newline{}SmartTrans NaviCore v3 modules development}{During this work I've developed:%
\begin{itemize}%
\item GPS/GLONASS telematic data from tracker/relay receiving servers;
\item telematic data filtering modules;
\item \texttt{bash} scripts to manage pipelined group of processes as single instance in a way like Linux daemons;
\item unified modules interface to have ability to use as standalone applications;
\item component watchdog.
\end{itemize} %
Also, I have implemented wrapper libraries that are used in product development implement: %
\begin{itemize}
\item asynchronous processing employing \texttt{epoll} and \texttt{eventfd} mechanisms;
\item thread pool;
\item \texttt{TCP/IP} connection and asynchronous data receive/transmit;
\item coroutines.
\end{itemize} %
During this work I've had experience with:%
\begin{itemize}
\item libraries: \texttt{Boost.Asio}, \texttt{libmicrohttpd}, \texttt{cpp-netlib}, \texttt{libpqxx}, \texttt{libsqlite3}, \texttt{libdl};
\item kernel subsystems: \texttt{epoll}, \texttt{UNIX-sockets};
\item shell scripting: \texttt{bash};
\item multithreading: concurrent and cooperative;
\item STL.
\end{itemize}}
\cventry{Jul 2013 -- Nov 2014}{Engineer}{Samsung R\&D Institute Ukraine}{Kharkov}{\newline{}System software development for Tizen OS}{\textbf{Responsibilities:}%
\begin{itemize}
\item writing system software code which does:use DBus interprocess communication, controls
processes, controls processes with systemd startup manager API;
\item writing middleware code that uses EFL GUI library;
\item writing middleware code for xserver input driver, modifying enlightenment window manager
input modules;
\item writing unit-tests and functional tests using check framework.
\end{itemize}}

\section{Languages}
\cvitemwithcomment{Russian}{Native}{}
\cvitemwithcomment{English}{Intermediate/Technical}{}

\section{About myself}
\par{I've had vast experience in:}
\begin{itemize}
\item \texttt{C} and \texttt{C++} development for GNU/Linux with CLI and \texttt{EFL} or \texttt{QT} GUI, including interprocess communications using \texttt{UNIX sockets}, \texttt{pipe}s and a little of \texttt{DBus} IPC;
\item \texttt{bash} shell scripting to run pipelined processes as some sort of daemon with status acquisition, start/stop/restart methods support with per-instance static configuration.
\end{itemize}
\par{I have experience with test-driven development.}
\par{While programming with C++ I've used \texttt{Boost.Asio}, \texttt{STL}, multithreaded and cooperative multitasking.}
\par{Also I've had experience developing embedded projects with Atmel's 8-bit MCU. These projects were intended to control a set of stepper motors and employed distributed logic to simulate air flow pressures in wind tunnel and simulate liquid flow. For both these projects I've developed both hardware and software.}
\par{I've participated at development of the system to monitor vehicles and theirs telemetry and raise an alarm or notification in case of some event (like fueling, drain, enter/exit geo-zone) with satellite navigation technologies (GPS/GLONASS). This required parsing of GPS/GLONASS trackers' protocols, implementing geo-mathematical algorithms, sensor data filtering, etc.}
\par{At the same time I've had moderate experience with PostgreSQL and SQLite 3 databases in this project.}
\par{Also, I was involved in development of application that used multigrids to prepare cartographic data for route-finding algorithms based on graph data structures.}
\par{Also, I'm interested in studying analogue and digital circuitry and ARM assembly as well as networking protocols and cryptography.}
\par{Also, I have a certificate (No 46694) of registration of copyright on the product:
\begin{itemize}
\item \textbf{Title:} The computer program ``Raster map images to multigrid converter with subsequent conversion into graph-based data structures''
\item \textbf{Authors:} Mikhnev Sergey S., Kulik Nikolay S., Kvasnikov Vladimir P., Kanaev Sergey V.
\item \textbf{Property rights after:} National Aviation University, 1 Komarov Ave, Kiev, 03680
\item \textbf{Registration date} 10.12.2012
\end{itemize}}
\par{Currently I'm studying distributed database design principles.}

\clearpage
\end{document}
